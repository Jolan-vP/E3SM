\documentclass{article}
\usepackage[utf8]{inputenc}
\usepackage{graphicx}
\usepackage{amsmath}
\usepackage{geometry}
\usepackage{pdfpages}
\usepackage{caption}
\usepackage{float}
\usepackage{url}
\usepackage{wrapfig}
\usepackage{color}
\usepackage{multicol}


\geometry{margin=0.75in}
\title{ME 555 Experimental Design and Research Methods \newline \textbf{Manufacturing Procedure File}}\\
\author{Jol\'an von Plutzner}

\begin{document}
\maketitle
%{\normalsize
\section{General Concept}
The manufacturing procedure consists of materials gathering, mold development, and layup. 
\par
\begin{multicols}{2}
\section{Materials}
\textbf{Carbon Fiber:}
\begin{itemize}
    \item[--] Cross-Weave, moderate tow
    \item[--] Uni-directional weave
    \item[--] Two Part Epoxy
    \item[--] Round Cutter 
    \item[--] P100 Respirator
\end{itemize}
\textbf{Layup Accessories:}
\begin{itemize}
    \item[--] Peel Ply
    \item[--] Vacuum Bag
    \item[--] Cardboard / protective work surface
    \item[--] Duct Tape 
    \item[--] Measuring Tape
    \item[--] Stir cups and spoons
    \item[--] Electronic Scale
\end{itemize}
\end{multicols}
\section{Procedure}
\subsection{Mold Development}
\begin{enumerate}
    \item Design interior molds in Solidworks such that the outer geometry is 2mm smaller in all directions. 3D print in ABS.
    \item Sand molds to eliminate large imperfections. 
    \item Apply acetone to smooth printed surface. 
    \item Layer with mold release wax applied with cloth. 
    \item Over wax layer apply 4 coats of PVA release fluid. Allow several minutes to dry between layering. 
\end{enumerate}
\subsection{Layup}
\begin{enumerate}
    \item Put on respirator and gloves, and put cardboard down on work surface. Setup camera and begin filming time lapse. 
    \item Test vacuum pump to ensure sufficient suction
    \item Carbon Fiber: Using the unfolded tube templates, cut the carbon fiber using the round cutter with 2cm to spare along the longest edges. Weigh each piece on scale. Handle these pieces as little as possible to prevent fraying.
        \begin{itemize}
            \item[--] Layers 1 and 2 for diamond tube. Layer 1 weight \underline{~~~~~~~~~~~~~~~~~~~~.} Layer 2 weight \underline{~~~~~~~~~~~~~~~~~~~~.}
            \item[--] Layers 1 and 2 for half-moon tube. Layer 1 weight \underline{~~~~~~~~~~~~~~~~~~~~.} Layer 2 weight \underline{~~~~~~~~~~~~~~~~~~~~.}
            \item[--] Two layers for each coupon card. Layer 1 weight \underline{~~~~~~~~~~~~~~~~~~~~.} Layer 2 weight \underline{~~~~~~~~~~~~~~~~~~~~.}
        \end{itemize}
    \item Peel Ply: Cut one sheet of peel ply per tube with about 3 inches to spare on each side \\
    Cut one peel ply cover for the three coupons
    \item Measure and cut tube bag to ensure that both tubes and coupons. Seal one end of tube bagging with duct tape. 
    \item Cut one breather cloth section for each tube and set aside. Cut another section of breather cloth for coupons. Design vacuum bridges and cut out, lay aside
    \item Gently lay out molds on cardboard with their corresponding layers. Lay flat the peel ply layers nearby 
    \item Calculate how much of parts A and B epoxy are needed. Measure out into cups and stir with spoon. 25 minute timer starts now. 
    \item Wrap first layer of CF around first mold and apply epoxy by hand such that the intersection line lines up with the center of the top surface. Ensure even coat by looking at various lighting conditions and angles. Try to ensure adequate weave cohesion at edge intersection. Replicate for second mold. 
    \item Apply second layer of weave around each mold with the intersection line at the center of the bottom surface. Again try for edge cohesion and smoothness. Place completed tubes within peel ply sheets and set aside.
    \item Lay two ply of the cut coupon cards if time allows. Place within peel ply.
    \item Wrap breather cloths around both tubes individually and place in the bag. Wrap breather cloth around coupon plys and put into vacuum bag.  
    \item Seal bag edge and insert vacuum tube. Turn on vacuum and pump out air. Let cure remaining 25 minutes. 
\end{enumerate}
\end{document}

